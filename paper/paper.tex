% Documento LaTeX com o arquigo que estamos escrevendo

%Cabeçalho
% % Onde a gente configura o documento

% Vamos definir a classe do artigo, e influencia na formatação
\documentclass{article} 

%Vamos importar um pacote
\usepackage[brazil]{babel}
\usepackage{graphicx}


%Corpo
% Onde a gente escreve o texto

%Começa o documento
\begin{document}

\title{Análise de variação de temperatura dos últimos cinco anos}
\author{Fernando Lima de Oliveira}
%Cria o título
\maketitle

\begin{abstract}
Meu resumo
\end{abstract}

Meu primeiro artigo em LaTeX

\section{Introdução}

Isso é a introdução
Outra frase da inrodução

Esse já vai ser outro parágrafo

\section{Metodologia}
\label{sec:metodos}

	Aqui vou descrever tudo o que eu fiz
Ajustamos uma reta aos cinco últimos anos dos dados
de temperatura média mensal para cada país.
Assim calculamos a taxaa de variação da temperatura recente

\begin{equation}
y = \int \Omega x dx
\end{equation}

A equação da reta é

\begin{equation}
T(t) = a t + b \ ,
\label{eq:reta}
\end{equation}

\noindent
Onde $T$ é a temperatura; 

\noindent
$t$ o tempo; 

\noindent
$a$ o coeficiente angular; 

\noindent
$b$ o coeficiente linear.

	Utilizamos a equação \ref{eq:reta} em um código Python para fazer o ajuste da reta com 
o Método dos Mínimos Quadrados (MMQ). Isso está descrito na seção \ref{sec:metodos}.

\section{Resultados}

	Analisamos os dados de 225 países. São muitos para listar aqui. Confia.
Os resultados da análise de temperatura estão na figura \ref{fig:variacao}

\begin{figure}[!htb]
	\centering
	\includegraphics[width=0.7\columnwidth]{../figuras/variacao_de_temperatura.png}
	\centering
	\caption{
		Variação de tempuratura média mensal dos últimos cinco anos
		a) Países com as cinco maiores variações de temperatura
		b)Países com as conco menores variações de temperatura.
	}
	\label{fig:variacao}
\end{figure}

\end{document}



